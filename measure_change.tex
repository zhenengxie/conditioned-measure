% !TEX program = xelatex

\documentclass{article}

\usepackage{amsmath, mathtools}

\title{Measure change under conditioning}
\author{}
\date{}

\newcommand{\R}{\mathbb R}
\newcommand{\N}{\mathbb N}
\newcommand{\littleo}{o}
\newcommand{\bigo}{O}

\begin{document}

\maketitle

\section{Introduction}

The configuration model is a well known method of sampling a uniform random graph with a given degree sequence. The technique works by sampling a uniform matching of half edges, then conditioning on the resulting multigraph being simple.

\section{Exact form of the measure change}

\section{Approximation of the measure change}

\subsection{Exponential tilting}

\subsection{Local limit theorems}

\end{document}